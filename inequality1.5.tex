\documentclass[a4paper]{report}
\usepackage[utf8]{inputenc}
%\usepackage{ctexcap}
\usepackage{titlesec,titletoc,multicol,titling}
\usepackage{amsmath,bm,latexsym}
\usepackage{amssymb,amsfonts,amsthm,mathrsfs}
\usepackage[nottoc]{tocbibind}
\usepackage{verbatim,cite}
\usepackage{indentfirst}
\usepackage{hyperref}
\usepackage{enumerate}
\usepackage{geometry}
\usepackage{mathtools}
%\usepackage{ntheorem}
\usepackage{enumerate}
%\usepackage{xcolor}
\usepackage{xfrac}
\usepackage{esint}
%\usepackage[cm-default]{fontspec} 
\usepackage{fancyhdr}
\usepackage[Lenny]{fncychap}
\usepackage{cmbright}
%\usepackage[utf8x]{inputenc}
%\usepackage[T1,T2A]{fontenc}
%\usepackage[T1]{fontenc}
%\usepackage{ccfonts,eulervm}
%\usepackage{listings}

\geometry{left=3.25cm,right=3.25cm,top=2.5cm,bottom=2.5cm}
\setlength{\parindent}{0pt}
%\setlength{\arraycolsep}{2.4pt}

%\usepackage{newunicodechar}
%\newunicodechar{00a0}{~}
%\DeclareUnicodeCharacter{00A0}{~}

%document
\begin{document}
	
\renewcommand{\arraystretch}{1.2}
\newcommand{\upcite}[1]{\textsuperscript{\cite{#1}}}

\newtheoremstyle{thr}{1.5ex plus 1ex minus .2ex}{1.5ex plus 1ex minus .2ex}{\rm}{\parindent}{\bfseries}{}{1em}{}
\theoremstyle{thr}
\newtheorem{thr}{Theorem}[section]
\newtheorem{myex}{Example}[section]
\newtheorem{defi}{Definition}[section]
\linespread{2} 

\title{\bf{INEQUALITY SUMMARY}}
\author{Hooy Chen}
\maketitle
\newpage

\begin{abstract}
	\setlength{\parindent}{0pt} \setlength{\parskip}{1.5ex plus 0.5ex
		minus 0.2ex} %\noindent
    This article summarizes some of common inequalities.
\end{abstract}
\newpage

\pdfbookmark[1]{Contents}{anchor}
\tableofcontents
\newpage

% Create a new 1st level heading

\chapter{Jensen's Inequality}
\section{Statement}
\begin{thr}
For a real convex function $f$, numbers $x_1,x_2,\cdots ,x_n$ in its domain, and positive weights $a_i$, Jensen's inequality can be stated as:
    \begin{equation}
    f\left(\frac{\sum a_ix_i}{\sum a_j}\right)\leqslant
    \frac{\sum a_if(x_i)}{\sum a_j}
\label{1.1}
    \end{equation}
and the inequality is reversed if $f$ is concave, which is
    \begin{equation}
    f\left(\frac{\sum a_ix_i}{\sum a_j}\right)\geqslant
    \frac{\sum a_if(x_i)}{\sum a_j}.
\label{1.2}
    \end{equation}
Equality holds if and only if $x_1=x_2=\cdots=x_n$ or $f$ is linear\upcite{DavidChandler}.\\
\end{thr}
As a particular case, if the weights $a_i$ are all equal, then 
(\ref{1.1}) and (\ref{1.2}) become  
    \begin{equation}
    f\left(\frac{\sum a_ix_i}{n}\right)\leqslant
    \frac{\sum a_if(x_i)}{n}
    \end{equation}
and
    \begin{equation}
    f\left(\frac{\sum a_ix_i}{n}\right)\geqslant
    \frac{\sum a_if(x_i)}{n}.
    \end{equation}      

For instance, if $\lambda_1$ and $\lambda_2$ are two arbitrary nonnegative real numbers such that $\lambda_1 + \lambda_2 = 1$ then convexity of $\varphi$ implies
\[\forall x_1, x_2 :\qquad \varphi \left(\lambda_1 x_1 +\lambda_2 x_2\right)\leqslant \lambda_1\varphi(x_1)+\lambda_2\varphi(x_2)\]\par\;
This can be easily generalized: if $\lambda_1,\lambda_2, \cdots , \lambda_n$ are nonnegative real numbers such that $\lambda_1 +\lambda_2+ \cdots + \lambda_n = 1$, then
\[\varphi \left( \lambda _{1}x_{1}+\lambda _{2}x_{2}+\cdots +\lambda _{n}x_n\right) \leqslant\lambda _{1}\varphi (x_{1})+\lambda _{2}\varphi (x_{2})+\cdots +\lambda_n\varphi(x_n)\]
for any $x_1, x_2, \cdots , x_n$. 


\begin{proof}[\bf{Proof}]
This finite form of the Jensen's inequality can be proved by induction: by convexity hypotheses, the statement is true for $n = 2$. Suppose it is true also for some $n$, one needs to prove it for $n+1$. At least one of the $\lambda_i$ is strictly positive, say $\lambda_1$ ;therefore by convexity inequality:
\[\varphi \left(\sum_{i=1}^{n+1} \lambda_i x_i \right) = \varphi\left(\lambda_1x_1+(1-\lambda _{1})\sum \limits^{n+1}_{i=2}\frac{\lambda _{i}}{1-\lambda _{1}} x_{i}\right)\]

\[\leqslant \lambda_1\varphi(x_1)+(1-\lambda _{1})\varphi\left(\sum \limits^{n+1}_{i=2}\frac{\lambda _{i}}{1-\lambda _{1}} x_{i}\right).\]
Since
\[\sum \limits^{n+1}_{i=2}\frac{\lambda _{i}}{1-\lambda _{1}} =1,\]
one can apply the induction hypotheses to the last term in the previous formula to obtain the result, namely the finite form of the Jensen's inequality \upcite{Rudin}.

\end{proof}


\chapter{Rearrangement Inequality} 
\section{Statement of the inequality}
\begin{thr}
	\begin{equation} 
	x_n y_1+\dots + x_1 y_n\leqslant x_{\sigma (1) } y_1+\dots + x_{\sigma (n)} y_n\leqslant x_1 y_1 +\dots +x_n y_n
	\end{equation}
	for every choice of real numbers
	\[x_1\leqslant \cdots\leqslant x_n \quad and \quad y_1\leqslant\cdots\leqslant y_n\]
	and every permutation
	\[x_{\sigma \left(1\right) } ,\dots , x_{\sigma (n)}\]
	of
	\[x_1,...,x_n\ .\]\\
\end{thr}
If the numbers are different, meaning that
\[x_1< \cdots <x_n \quad and \quad y_1<\cdots <y_n\ ,\]
then the lower bound is attained only for the permutation which reverses the order, i.e.,
$\sigma (i) = n -i +1$ for all $\;i = 1, ... , n$, and the upper bound is attained only for the identity, i.e., $\sigma (i) = i$ for all $\;i = 1, ... , n$.

Note that the rearrangement inequality makes no assumptions on the signs of the real numbers.\upcite{Hardy}
\par\;\par\;

\section{Generalization}
\begin{thr}

\begin{equation}
\sum \limits^{n}_{j=1}\prod \limits^{m}_{i=1}a^{\,\prime}_{ij}\leqslant
\sum \limits^{n}_{j=1}\prod \limits^{m}_{i=1}a_{ij},\label{1}
\end{equation}

\begin{equation}
\prod \limits^{n}_{j=1}\sum \limits^{m}_{i=1}a^{\,\prime}_{ij}\geqslant
\prod \limits^{n}_{j=1}\sum \limits^{m}_{i=1}a_{ij}\label{2}
\end{equation}
for every choice of real numbers
	$0\leqslant a_{i1}\leqslant a_{i2} \cdots\leqslant a_{in} $
	and every permutation
	$a^{\,\prime}_{i1}\leqslant a^{\,\prime}_{i2} \cdots\leqslant a^{\,\prime}_{in} $
	of	$a_{i1},a_{i2},\cdots,a_{in}\ (i=1,2,\cdots,m).$
\end{thr}
(\ref{1}) and (\ref{2}) are called \emph{S inequality} and \emph{T inequality} in short.
To help understand and apply the two inequalities, we consider two 
matrixes arranged by $m\times n$ numbers ($a^{\,\prime}_{i1}\leqslant a^{\,\prime}_{i2} \cdots\leqslant a^{\,\prime}_{in} ,
	 i=1,2,\cdots,m$):

\[A=
\left(\begin{array}{cccc}
a_{11}& a_{12}& \cdots & a_{1n} \\
a_{21}& a_{22}& \cdots & a_{2n} \\
\vdots& \vdots & & \vdots \\
a_{n1}& a_{n2}& \cdots & a_{nn}
\end{array}\right),
\]

\[A^{\,\prime}=
\left(\begin{array}{cccc}
a^{\,\prime}_{11}& a^{\,\prime}_{12}& \cdots & a^{\,\prime}_{1n} \\
a^{\,\prime}_{21}& a^{\,\prime}_{22}& \cdots & a^{\,\prime}_{2n} \\
\vdots& \vdots & & \vdots \\
a^{\,\prime}_{n1}& a^{\,\prime}_{n2}& \cdots & a^{\,\prime}_{nn}
\end{array}\right),
\]

\chapter{Schur's Inequality}
\section{Statement}
\begin{thr}
For all non-negative real numbers $x, y, z$ and a positive number $t$,
\begin{equation}
x^{t}(x-y)(x-z)+y^{t}(y-z)(y-x)+z^{t}(z-x)(z-y)\geqslant0
\end{equation}
with equality if and only if $x = y = z$ or two of them are equal and the other is zero.\\ 
\end{thr}

When $t=1$, the following well-known special case can be derived:

\begin{equation}
x^3+y^3+z^3+3xyz\geqslant xy(x+y)+xz(x+z)+yz(y+z).
\end{equation}

\begin{proof}[\bf{Proof}]

Since the inequality is symmetric in $x,y,z$. we may assume without loss of generality that $x\geqslant y\geqslant z$. Then the inequality
\[(x-y)[x^{t}(x-z)-y^{t}(y-z)]+z^{t}(x-z)(y-z)\geqslant 0\]
clearly holds, since every term on the left-hand side of the equation is non-negative. This rearranges to Schur's inequality.
\end{proof}

\chapter{Average Value Inequality}

    \section{The basic form of the inequality of average value}

	\begin{thr}
	
	Let $a_i>0$ , $i=1,2,$ \dots $,n.$
	
		\begin{equation}
		\sqrt[n]{\frac{a_1^2+a_2^2+\dots+a_n^2}{n}}
		\geqslant\frac{a_1+a_2+\dots+a_n}{n}
		 \geqslant\sqrt{a_1a_2\dots a_n}
		\geqslant\frac{n}{\frac{1}{a_1}+\frac{1}{a_2}+\dots +\frac{1}{a_n}},
		\end{equation}
	    Equality holds if and only if
$a_1=a_2=\cdots=a_n=0$\upcite{Steele}.
    
	    The above inequality is often written in this form
	    \[\bf{Q_n}\geqslant
	    \bf{A_n}\geqslant
	    \bf{G_n}\geqslant
	    \bf{H_n}\ .\]
	    \\
	\end{thr}
	\par 
	
	\begin{myex}
	
	Assume $n\in\mathbb{N}^+,x\in\mathbb{R}^+$.Prove that
	\[x+\frac{1}{nx^n}\geqslant\frac{n+1}{n}\ .\]
	
    \end{myex}

\begin{proof}[\bf{Proof}]
	\ From$\ \mathbf{A_n}\geqslant \mathbf{G_n}, $ we have
	\[\frac{a_1+a_2+\dots+a_{n+1}}{n+1}\geqslant\sqrt[n+1]{a_1a_2\dots a_{n+1}}\  ,\]
	\[x+\frac{1}{nx^n}=\underbrace{\frac{x}{n}+\cdots +\frac{x}{n}}_{n\ numbers}+\frac{1}{nx^n}
	\geqslant (n+1)\sqrt[n+1]{\frac{1}{n^{n+1}}}
	=\frac{n+1}{n}\ . \]
\end{proof}	
	
    
    \par  \ \par  \ 
    
    \section{AM-GM inequality}
	
	
	\begin{defi}
		
	Assume $a=(a_1,a_2,\dots ,a_n)$,
	$\ a_k\geqslant 0$
	,$1\leqslant k\leqslant n$.
	$\mathbf{A_n}=\dfrac{1}{n}\sum\limits_{k=1}^n a_k\quad$,
	called the \emph{Arithmetic Mean} of
	$\ a_1,a_2,\dots ,a_n\ .\ 
	\mathbf{G_n}=(\prod\limits_{k=1}^n a_k)^\frac{1}{n}\quad$,
	called the \emph{Geometric Mean} of 
	$\ a_1,a_2,\dots ,a_n$\upcite{Cauchy}.
	
	\end{defi}
	\par  \ 

\begin{thr}
	\begin{equation}
	\mathbf{A_n}\geqslant \mathbf{G_n}	
	\end{equation}
	
	or
	
	\begin{equation}
	\left( \ \frac{1}{n}\sum_{k=1}^n a_k \right)^n\quad\geqslant\quad \prod\limits_{k=1}^n a_k \  .
    \end{equation}
\end{thr}

	
\begin{proof}[\bf{Proof}]
Note that geometrical mean $\mathbf{G_n}$ is actually equal to 
\[ \exp \left(\frac{\text{ln}x_{1}+\text{ln}x_{2}+\cdots +\text{ln}x_{n}}{n} \right),\]
Since the natural logarithm is strictly increasing, AM-GM inequality is equivalent to
\[\ln \frac{x_{1}+x_{2}+\cdots +x_{n}}{n} \geqslant \frac{\text{ln}x_{1}+\text{ln}x_{2}+\cdots +\text{ln}x_{n}}{n} .\]
Since the natural logarithmic function is strictly concave, according to Jensen's inequality we can imply that the above inequality holds. We have thus proved the theorem.

\end{proof}
	\par  \ 

	\begin{thr}
		
		\begin{equation}
		\mathbf{G_n}(a,q)\leqslant \mathbf{A_n}(a,q)\ ,
		\end{equation}
    where
    \[\mathbf{G_n}(a,q)=\prod_{k=1}^{n} a_k^{q_k}\ ,\ 
	\mathbf{A_n}(a,q)=\sum_{k=1}^{n} q_k a_k\ ,\ 
    q_k>0\ ,\ \sum_{k=1}^{n} q_k =1  ,\]
	in other words,
    \[a_1^{q_1} a_2^{q_2}\cdots a_n^{q_n}
    \leqslant a_1q_1+a_2q_2+\dots +a_nq_n
    \ ,\ q_1+q_2+\dots +q_n=1
    \ .\]
       	
    \end{thr}
    $\mathbf{G_n}$ and $\mathbf{A_n}$ can be connected by logarithmic transformation . 
    Suppose $\ln a=(\ln a_1,\ln a_2,\dots ,\ln a_n)$,
    thus \[\ln \mathbf{G_n}(a,q)=\ln a_1^{q_1} a_2^{q_2}
    \cdots a_n^{q_n}=\sum_{k=1}^n q_k \ln a_k=\mathbf{A_n}(\ln a,q)\;.\]
    
	
	Formula (3) can be generalized.\ Let $ a_{jk} > 0,\ q_k >0,\ and
	\sum\limits_{k=1}^n q_k = 1\ $,
	then
	
	      \begin{equation}
	      \sum_{j=1}^{m}\left(\prod_{k=1}^{m} a_{jk}^{q_k} \right)\leqslant
	     \prod_{k=1}^{m}\Bigg(\sum_{j=1}^{m} a_{jk} \Bigg)^{q_k}
	      \end{equation}

\par  \quad

\begin{myex}
Let $a,b,c,d >0$. Prove that
\[\frac{a}{b+c} +\frac{b}{c+d} +\frac{c}{d+a} +\frac{d}{a+b} \geqslant 2.\]
\end{myex}
\begin{proof}[\bf{Proof}]
	\[\frac{a}{b+c} +\frac{b}{c+d} +\frac{c}{d+a} +\frac{d}{a+b} \geqslant 2\]
    \[\iff \frac{2a+b+c}{b+c} +\frac{2b+c+d}{c+d} +\frac{2c+d+a}{d+a} +\frac{2d+a+b}{a+b} \geqslant 8\]
    \begin{equation}
\iff \left(\frac{a+b}{b+c} +\frac{b+c}{c+d}+\frac{c+d}{d+a}+\frac{d+a}{a+b}\right)+\left(\frac{a+c}{b+c} +\frac{c+a}{d+a}\right) +\left(\frac{b+d}{c+d}+\frac{d+b}{a+b}\right)  \geqslant 8.\tag{a}
\end{equation}
    From $\mathbf{A_n}\geqslant \mathbf{G_n}$, we obtain
    \begin{equation}
\left(\frac{a+b}{b+c} +\frac{b+c}{c+d}+\frac{c+d}{d+a}+\frac{d+a}{a+b}\right)\geqslant 4\;\sqrt[4]{\frac{a+b}{b+c}\cdot\frac{b+c}{c+d}\cdot\frac{c+d}{d+a}\cdot\frac{d+a}{a+b}}=4\tag{b}
\end{equation}
and
\[\left(a+b+c+d\right)\left(\frac{1}{b+c}+\frac{1}{d+a}\right)=\frac{a+d}{b+c}+\frac{b+c}{a+d}+2\geqslant2\sqrt{\frac{a+d}{b+c}\cdot\frac{b+c}{a+d}}+2=4\]
    \begin{equation}
\iff\frac{a+c}{b+c} +\frac{c+a}{d+a}\geqslant\frac{4\left(a+c\right)}{a+b+c+d}.\tag{c}
\end{equation}
    In a similar way, we have
    \begin{equation}
\frac{b+d}{c+d}+\frac{d+b}{a+b}\geqslant\frac{4\left(b+d\right)}{a+b+c+d}.\tag{d}
   \end{equation}
Add (b), (c), (d) up and we get (a). This completes
 the proof.
\end{proof}

    \par  \ \par \ 
\section{RMS-AM Inequality}
\begin{thr}
Root mean square-arithmetic mean inequality states that for positive numbers $x_1,x_2,\cdots,x_n$,
\begin{equation}	\sqrt[n]{\frac{a_1^2+a_2^2+\dots+a_n^2}{n}}
		\geqslant\frac{a_1+a_2+\dots+a_n}{n}.
\end{equation}
\end{thr}
\par\;\par

\section{Power Mean Inequality}
\begin{defi}
If $p$ is a non-zero real number, we can define the \emph{generalized mean} or \emph{power mean} with exponent $p$ of the positive real numbers $x_1, x_2,\cdots, x_n$ as:
\[
M_p(x_1, x_2,\cdots, x_n)=\left( \frac{x^{p}_{1}+x^{p}_{2}+\cdots +x^{p}_{n}}{n} \right) ^{\frac{1}{p} }.
\]
Furthermore, for a sequence of positive weights $w_i$ with sum $\sum a_i =1$, we define the \emph{weighted power mean} as:
\[
M_{p}(x_{1},x_{2},\cdots ,x_{n}) =\left( w_{1}x^{p}_{1}+w_{2}x^{p}_{2}+\cdots +w_{n}x^{p}_{n}\right) ^{\frac{1}{p} }.
\]
The unweighted means correspond to setting all $w_i= \frac{1}{n}$.
\end{defi}
\par \quad
\begin{thr}
In general, if $p\le q$, then 
\begin{equation}
M_p(x_1, x_2,\cdots, x_n) \leqslant
M_q(x_1, x_2,\cdots, x_n) ,
\end{equation}
and the two means are equal if and only if $x_1= x_2= \cdots = x_n$.
\end{thr}

If $p\ge 1$, then
\[\frac{x^{p}_{1}+x^{p}_{2}+\cdots +x^{p}_{n}}{n}\geqslant
\left( \frac{x_{1}+x_{2}+\cdots +x_{n}}{n} \right) ^{p}.\]

\begin{proof}[\bf{Proof}]
It follows from the fact that, for all real $p$,
\[\frac{\partial }{\partial p} M_{p}(x_1, x_2,\cdots, x_n)\geqslant0,\]
which can be proved using Jensen's inequality.
\end{proof}


\chapter{Mahler's Inequality}
\section{Statement}
\begin{thr}
Mahler's inequality, named after Kurt Mahler, states that the geometric mean of the term-by-term sum of two finite sequences of positive numbers is greater than or equal to the sum of their two separate geometric means:
\begin{equation}
\prod_{i=1}^n (x_i+y_i)^{\frac{1}{n}}\geqslant\prod_{i=1}^nx_i^{\frac{1}{n}}+\prod_{i=1}^ny_i^{\frac{1}{n}},
\end{equation}
when $x_i, y_i > 0$ for all $k$\upcite{Bullen}.
\end{thr}


\begin{proof}[\bf{Proof}]
By the inequality of arithmetic and geometric means, we have:
\[\prod_{i=1}^n\left( \frac{x_i}{x_i+y_i}\right)^{\frac{1}{n}}
\leqslant\frac{1}{n}\sum_{i=1}^n\left( \frac{x_i}{x_i+y_i}\right)\]
and
\[\prod_{i=1}^n\left( \frac{y_i}{x_i+y_i}\right)^{\frac{1}{n}}
\leqslant\frac{1}{n}\sum_{i=1}^n\left( \frac{y_i}{x_i+y_i}\right).\]
Hence,
\[\prod_{i=1}^n\left( \frac{x_i}{x_i+y_i}\right)^{\frac{1}{n}}+
\prod_{i=1}^n\left( \frac{y_i}{x_i+y_i}\right)^{\frac{1}{n}}\leqslant
\frac{1}{n} n=1 .\]
Clearing denominators then gives the desired result.
\end{proof}


\chapter{Cauchy–Schwarz Inequality}
	\section{Statement}

\begin{thr}

Let$\ a_i,\:b_i\in \mathbb{R},\;i=1,2,...,n.$
\begin{equation}
\left(\sum_{i=1}^n a_i^2\right)\;\left(\sum_{i=1}^n b_i^2\right) \;\geqslant
\left(\sum_{i=1}^n a_i b_i\right) ^2 \ , 
\end{equation}
Equality holds if and only if$\quad a_i=\lambda b_i\;$ ($\lambda$ is a constant).\upcite {Bityutskov} \par  
\end{thr}    


\begin{proof}[\bf{Proof}]
We can check that
\[\left(\sum_{i=1}^na_i^2\right)\left(\sum_{i=1}^nb_i^2\right) -
\left(\sum_{i=1}^na_ib_i\right)^2 =\ \sum_{1\leqslant i<j\leqslant n}(a_ib_j-a_jb_i)^2 \geqslant 0 .\]
 It is now obvious that the theorem holds.
\end{proof}
\par \;

\section{Application}
\begin{myex}
	
	Prove that for all positive $x,y$ and $z$, we have
	\[x+y+z\leqslant 2\left(\frac{x^2}{y+z}+\frac{y^2}{x+z}+\frac{z^2}{x+y}\right).
	\]
\end{myex}	

\begin{proof}[\bf{Proof}]
Indeed,	\[(x+y+z)^2=\left(x\sqrt{\frac{y+z}{y+z}}+y\sqrt{\frac{x+z}{x+z}}+z\sqrt{\frac{x+y}{x+y}}\;\right)^2\]
\[\leqslant \left(\frac{x^2}{y+z}+\frac{y^2}{x+z}+\frac{z^2}{x+y}\right)(y+z+x+z+x+y)
	\]
\[\iff x+y+z\leqslant 2\left(\frac{x^2}{y+z}+\frac{y^2}{x+z}+\frac{z^2}{x+y}\right).\]
	An alternative proof is simply obtained with the Jensen inequality.\\
	The inequality is
	\[f\left(\frac{x+y+z}{3}\right)\leqslant\frac{f(x)+f(y)+f(z)}{3}
	\]
	for all positive$\ x$,$y$,$z$ with sum $A$, where $\displaystyle f(u)=\frac{u^2}{A-u}$, and $A=x+y+z$.
	This is equivalent to $f(u)$ being convex in the interval $(0,A)$.The second derivative is\[f''(x) =\frac{2A^2}{(A-u)^3}\geqslant 0 . \]
	\end{proof}
\par \ 

\begin{myex}
Let $a_i>0 (i=1,2,\dots,n)$,  and $\sum\limits_{i=1}^n a_i=1$.Then\[\sum \limits^{n}_{i=1}\left(a_{i}+\frac{1}{a_{i}}
\right)^{2}\geqslant\frac{(1+n^{\,2})^{2}}{n} .\]
\end{myex}

\begin{proof}[\bf{Proof}]
\[(1^{2}+1^{2}+ \cdots+1^{2})\sum \limits^{n}_{i=1}\left( a_{i}+\frac{1}{a_{i}} \right) ^{2}\]
\[\geqslant\left[ \left( a_{1}+\frac{1}{a_{1}}  \right) +\left( a_{2}+\frac{1}{a_{2}} \right)+\cdots+ \left( a_{n}+\frac{1}{a_{n}} \right)\right]^2 \]
\[=\left(\sum^n_{i=1}a_i+\sum_{i=1}^n\frac{1}{a_i}\right)^2.\]
Note that\[\sum^n_{i=1}a_i=1,  \;\; \left(\sum^n_{i=1}a_i\right)\cdot\left(\sum_{i=1}^n\frac{1}{a_i}\right)\geqslant n^{2}.\]
Thus, we can infer that
\[\sum \limits^{n}_{i=1}\left(a_{i}+\frac{1}{a_{i}}
\right)^{2}\geqslant\frac{(1+n^{\,2})^{2}}{n} .\]
\end{proof}
\par \quad

\begin{myex}
Let $a_1, a_2, \dots,a_n\in\mathbb{R}^+$,$n\in\mathbb{N}^+$,$n\geqslant2$.Then \[\frac{s}{s-a_{1}} +\frac{s}{s-a_{2}} +\cdots +\frac{s}{s-a_{n}}\geqslant \frac{n^{\,2}}{n-1} ,\]where $s=a_1+a_2+\dots+a_n$.
\end{myex}

\begin{proof}[\bf{Proof}]
Notice that\[(n-1)s=ns-(s=a_1+a_2+\dots+a_n)=(s-a_1)+(s-a_2)+\dots+(s-a_n).\]
According to Cauchy–Schwarz inequality, it follows\[\left[(s-a_1)+(s-a_2)+\dots+(s-a_n)\right]\left(\frac{1}{s-a_{1}} +\frac{1}{s-a_{2}} +\cdots +\frac{1}{s-a_{n}}\right)\]
\[\geqslant\left(\sqrt{s-a_1}\cdot \frac{1}{\sqrt{s-a_ 1} } +\sqrt{s-a_ 2}\cdot \frac{1}{\sqrt{s-a_ 2} } +\cdots +\sqrt{s-a_ n} \cdot\frac{1}{\sqrt{s-a_ n} } \right)^2=n^2.\]
Hence,\[s(n-1)\left(\frac{1}{s-a_{1}} +\frac{1}{s-a_{2}} +\cdots +\frac{1}{s-a_{n}}\right)\geqslant n^2.\]
Thus we have derived that\[\frac{s}{s-a_{1}} +\frac{s}{s-a_{2}} +\cdots +\frac{s}{s-a_{n}}\geqslant \frac{n^{\,2}}{n-1} .\]
\end{proof}

\section{Corollary}
\begin{thr}
\begin{equation}
\left| \;\sqrt{\sum \limits^{n}_{i=1}a^{2}_{i}} -\sqrt{\sum \limits^{n}_{i=1}b^{2}_{i}} \;\right| \leqslant \sqrt{\sum \limits^{n}_{i=1}(a_{i}-b_{i})^{2}\;}.
\end{equation}
\end{thr}

\begin{proof}[\bf{Proof}]
According to Cauchy inequality, we have
\[\sqrt{\sum \limits^{n}_{i=1}a^{2}_{i}\sum \limits^{n}_{i=1}b^{2}_{i}} \geqslant\sum \limits^{n}_{i=1}a_{i}b_{i}.\]
Then
\[\sum\limits^{n}_{i=1}a^{2}_{i}
-2\sqrt{\sum \limits^{n}_{i=1}a^{2}_{i}\sum \limits^{n}_{i=1}b^{2}_{i}} 
+\sum \limits^{n}_{i=1}b^{2}_{i}
\leqslant
\sum\limits^{n}_{i=1}a^{2}_{i}
-2\sum \limits^{n}_{i=1}a_{i}b_{i}
+\sum\limits^{n}_{i=1}b^{2}_{i},\]


\end{proof}

\chapter{Chebyshev's Sum Inequality}
\section{Statement}
\begin{thr}

 If
\[a_1\geqslant a_2\geqslant\cdots\geqslant a_n\]
and
\[b_1\geqslant b_2\geqslant\cdots\geqslant b_n,\]
then
\begin{equation}
  \frac{1}{n}\sum_{k=1}^{n}a_k\cdot b_k\geqslant
  \left(\frac{1}{n}\sum_{k=1}^{n}a_k\right)
  \left(\frac{1}{n}\sum_{k=1}^{n}b_k\right)\; .
\end{equation}  

Similarly, if 
\[a_1\geqslant a_2\geqslant\cdots\geqslant a_n\]
and
\[b_1\leqslant b_2\leqslant\cdots\leqslant b_n,\]
then
\begin{equation}
\frac{1}{n}\sum_{k=1}^{n}a_k\cdot b_k\leqslant
\left(\frac{1}{n}\sum_{k=1}^{n}a_k\right)
\left(\frac{1}{n}\sum_{k=1}^{n}b_k\right)\; .
\end{equation} 

\end{thr}
\begin{proof}[\bf{Proof}]
	                                       
	Consider the sum
	\[S=\sum_{j=1}^n\sum_{k=1}^{n}(a_j-a_k)(b_j-b_k).\]
	The two sequences are non-increasing, therefore $a_j-a_k$ and $b_j-b_k$ 
	have the same sign for any $j, k$. Hence $S\geqslant0$.
	Opening the brackets,we reduce:
	\[0\leqslant2n\sum_{j=1}^{n}a_jb_j-2\sum_{j=1}^{n}a_j\sum_{k=1}^{n}b_k,\]
	whence
	\[\frac{1}{n}\sum_{k=1}^{n}a_k\cdot b_k\geqslant
	\left(\frac{1}{n}\sum_{k=1}^{n}a_k\right)
	\left(\frac{1}{n}\sum_{k=1}^{n}b_k\right)\; .\]
	
\end{proof}


\chapter{Young's Inequality}
\section{Statement}
\begin{thr}
If $a$ and $b$ are nonnegative real numbers and $p$ and $q$ are positive real numbers such that $\frac{1}{p}+\frac{1}{q}=1$, then
\begin{equation}
ab\leqslant\frac{a^p}{p}+\frac{b^q}{q}.
\end{equation}
Equality holds if and only if $\displaystyle a^p=b^q$.
\end{thr}

\begin{proof}[\bf{Proof}]

The claim is certainly true if  $a = 0$ or $b = 0$. Therefore, assume $a>0$ and $b>0$ in the following.  Put $t =\frac{1}{p}$, and $(1-t) = \frac{1}{q}$. Then since the logarithmfunction is strictly concave
\[\log \left(t a^p +(1-t) b^q \right) \geqslant t \log  \left( a^p\right) + (1-t) \log\left(b^q\right)=\log(a)+\log(b)+w\]
with equality if and only if $a^p = b^q$. Young's inequality follows by exponentiating.
\end{proof}

This form of Young's inequality is a special case of the inequality of weighted arithmetic and geometric means and can be used to prove H\"{o}lder's inequality.


\chapter{H\"{o}lder's Inequality}
\section{Statement}
    \begin{thr}
    	Let real numbers $a_i,b_i\geqslant0(i=1,2,\cdots ,n)$, $p,q\in\mathbb{R}\backslash\{0\}$ and $\displaystyle\frac{1}{p}+\frac{1}{q}=1$.\linebreak If $p>1$, then
    	\begin{equation}
    	\sum_{i=1}^{n}a_ib_i\leqslant
    	\left(\sum_{i=1}^{n}a_i^p\right)^{\frac{1}{p}}
    	\left(\sum_{i=1}^{n}b_i^q\right)^{\frac{1}{q}},
    	\end{equation}  
     and if $p<1$ and $p\neq0$, then
    	\begin{equation}
    	\sum_{i=1}^{n}a_ib_i\geqslant
    	\left(\sum_{i=1}^{n}a_i^p\right)^{\frac{1}{p}}
    	\left(\sum_{i=1}^{n}b_i^q\right)^{\frac{1}{q}},
    	\end{equation} 
with equality if and only if $\alpha a_i^p=\beta b_i^q$, where $i=1,2,\dotso ,n$, $\alpha^2+\beta^2\neq0$.
    \end{thr}
\par\;
\section{Application}
   \begin{myex}
   Let $a_i,b_i,m>0$ .We have
  \begin{equation}
  \sum_{i=1}^n \frac{a_i^{m+1}}{b_i^m}\geqslant\frac{\left(\displaystyle\sum_{i=1}^n a_i\right)^{m+1}}{\left(\displaystyle\sum_{i=1}^n b_i\right)^{m}},
  \end{equation} 
with equality if and only if $a_i=\lambda b_i$.
\end{myex}

\begin{proof}[\bf{Proof}]
In formula(15), assume $p=m+1$ with $m>0$, whence we get
\[\sum_{i=1}^na_ib_i\leqslant
\left(\sum_{i=1}^na_i^{m+1}\right)^{\frac{1}{m+1}}
\left(\sum_{i=1}^nb_i^{\frac{m+1}{m}}\right)^{\frac{m}{m+1}}.\] Then let $a_i=\displaystyle\frac{a_i}{b_i^{\frac{m}{m+1}}}, b_i=b_i^{\frac{m}{m+1}}$, so the above Inquality becomes
\[\sum_{i=1}^na_i\leqslant
\left(\sum_{i=1}^n\frac{a_i^{m+1}}{b_i^m}\right)^{\frac{1}{m+1}}
\left(\sum_{i=1}^n b_i\right)^{\frac{m}{m+1}}.\]
Rewriting it we obtain
  \[\sum_{i=1}^n \frac{a_i^{m+1}}{b_i^m}\geqslant\frac{\left(\displaystyle\sum_{i=1}^n a_i\right)^{m+1}}{\left(\displaystyle\sum_{i=1}^n b_i\right)^{m}}.\]
\end{proof}

\chapter{Minkowski Inequality}
\section{Statement}
\begin{thr}
If $a_k\geqslant0$, $b_k\geqslant0$,$ k=1,2,\cdots,n$, and $p\ge 1$, then
\begin{equation}
\left( \sum \limits^{n}_{k=1}\left( a_{k}+b_{k}\right) ^{p}\right) ^{\frac{1}{p} }\leqslant \left( \sum \limits^{n}_{k=1}a^{p}_{k}\right) ^{\frac{1}{p} }+\left( \sum \limits^{n}_{k=1}b^{p}_{k}\right) ^{\frac{1}{p} },\label{Minkowski}
\end{equation}
with equality if and only if $a_k=\lambda b_k$.
\end{thr}

\begin{proof}[\bf{Proof}]
\[
\sum \limits^{n}_{k=1}\left( a_{k}+b_{k}\right) ^{p}=\sum \limits^{n}_{k=1}a_{k}\left( a_{k}+b_{k}\right) ^{p-1}+\sum \limits^{n}_{k=1}b_{k}\left( a_{k}+b_{k}\right) ^{p-1}.\]
According to H\"{o}lder's inequality, since $\frac{1}{p} +\frac{1}{q} =1$, $p>1$, we have
\[\sum \limits^{n}_{k=1}a_{k}\left( a_{k}+b_{k}\right) ^{p-1}\leqslant \left( \sum \limits^{n}_{k=1}a^{p}_{k}\right) ^{\frac{1}{p} }\left( \sum \limits^{n}_{k=1}\left( a_{k}+b_{k}\right) ^{q\left( p-1\right) }\right) ^{\frac{1}{q} },
\tag{a}\]
\[\sum \limits^{n}_{k=1}b_{k}\left( a_{k}+b_{k}\right) ^{p-1}\leqslant \left( \sum \limits^{n}_{k=1}b^{p}_{k}\right) ^{\frac{1}{p} }\left( \sum \limits^{n}_{k=1}\left( a_{k}+b_{k}\right) ^{q\left( p-1\right) }\right) ^{\frac{1}{q} }.
\tag{b}\]
Add (a) to (b) and we have
\[\sum \limits^{n}_{k=1}\left( a_{k}+b_{k}\right) ^{p}\leqslant\left(\left( \sum \limits^{n}_{k=1}a^{p}_{k}\right) ^{\frac{1}{p} }+\left( \sum \limits^{n}_{k=1}b^{p}_{k}\right) ^{\frac{1}{p} }\right)\left( \sum \limits^{n}_{k=1}\left( a_{k}+b_{k}\right) ^{p}\right) ^{\frac{1}{q} }.\]
Divide the inequality by $\left( \sum \limits^{n}_{k=1}\left( a_{k}+b_{k}\right) ^{p}\right) ^{\frac{1}{q} }$, thus we obtain (\ref{Minkowski}).
\end{proof}

\chapter{Bernoulli's inequality}
\section{Statement}
\begin{thr}
If $x> -1$, then
   \begin{equation}
   (1+x)^r\geqslant 1+rx
   \end{equation}
for $r\geqslant 1$ or $r\leqslant 0$, and
   \begin{equation}
   (1+x)^r\leqslant 1+rx
   \end{equation}
for $0\leqslant r\leqslant 1$, with equality if and only if $x=0$ or $r=0,1$.\upcite{Carothers}
\end{thr}

\newpage
\begin{thebibliography}{}

\bibitem{DavidChandler}
David Chandler (1987). Introduction to Modern Statistical Mechanics. Oxford. ISBN 0-19-504277-8.

\bibitem{Rudin}
Walter Rudin(1987). Real and Complex Analysis. McGraw-Hill. ISBN 0-07-054234-1.

\bibitem{Steele}
Steele, J. Michael (2004). The Cauchy-Schwarz Master Class: An Introduction to the Art of Mathematical Inequalities. MAA Problem Books Series. Cambridge University Press. ISBN 978-0-521-54677-5. OCLC 54079548

\bibitem{Cauchy}
Cauchy, Augustin-Louis (1821). Cours d'analyse de l'École Royale Polytechnique, première partie, Analyse algébrique, Paris. The proof of the inequality of arithmetic and geometric means can be found on pages 457ff.

\bibitem{Bullen}
P.S. Bullen, D.S. Mitrinović, P.M. Vasić, "Means and their inequalities" , Reidel (1988)


\bibitem{Bityutskov}
Bityutskov, V.I. (2001), "Bunyakovskii inequality", in Hazewinkel, Michiel, Encyclopedia of Mathematics, Springer, ISBN 978-1-55608-010-4.

\bibitem{Hardy}
Hardy, G.H.; Littlewood, J.E.; Pólya, G. (1952), Inequalities, Cambridge Mathematical Library (2. ed.), Cambridge: Cambridge University Press, ISBN 0-521-05206-8, MR 0046395, Zbl 0047.05302, Section 10.2, Theorem 368.

\bibitem{Carothers}
Carothers, N. (2000). Real Analysis. Cambridge: Cambridge University Press. p. 9. ISBN 978-0-521-49756-5.


\end{thebibliography}

 
\end{document}